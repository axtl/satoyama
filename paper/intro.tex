\section{\label{sec:intro}Introduction}

Web applications have grown to be one of the most popular types of software in use today. As the World Wide Web provides a platform that is accessible to users regardless of geographic location or operating system, the issues that have previously hindered software distribution are no longer relevant. Instead, rapid development cycles and low economic risks, encourage software developers to adopt a web-centric view of product development. There is a significantly lower barrier to adoption, as no software must be downloaded, trusted to be free of viruses, or installed by users. Even in situations were native applications are subsequently developed, the availability of a web version is still desirable, as it serves both as a hub of activity for the user, and as a way to ensure data is accessible from anywhere, at anytime.

Rich user experiences on the web are increasingly developed with the aid of JavaScript(formally ECMAScript~\cite{ecmascriptiso}), and this fact has propelled the language to the upper ranks of various programming language indexes~\cite{ghstats}\cite{tlpi}\cite{tiobe}. These indexes may not be scientifically rigorous, yet they do offer insight into industry trends. Considering JavaScript's popularity and large developer base, its availability for server-side use makes it a very attractive target for learning. It is now possible to develop an application \emph{completely} in JavaScript.

This paper analyzes how such an end-to-end JavaScript application compares to a more traditional design. Section~\ref{sec:relwork} looks at related work. Section~\ref{sec:arch} provides details on the two architectures used for testing, especially the design. Section~\ref{sec:impl} provides details on the implementations, while section~\ref{sec:eval} evaluates them. Section~\ref{sec:fw} looks at possible avenues for improvement, while section~\ref{sec:con} concludes.